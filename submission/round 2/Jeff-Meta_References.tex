\documentclass[english,man]{apa6}

\usepackage{amssymb,amsmath}
\usepackage{ifxetex,ifluatex}
\usepackage{fixltx2e} % provides \textsubscript
\ifnum 0\ifxetex 1\fi\ifluatex 1\fi=0 % if pdftex
  \usepackage[T1]{fontenc}
  \usepackage[utf8]{inputenc}
\else % if luatex or xelatex
  \ifxetex
    \usepackage{mathspec}
    \usepackage{xltxtra,xunicode}
  \else
    \usepackage{fontspec}
  \fi
  \defaultfontfeatures{Mapping=tex-text,Scale=MatchLowercase}
  \newcommand{\euro}{€}
\fi
% use upquote if available, for straight quotes in verbatim environments
\IfFileExists{upquote.sty}{\usepackage{upquote}}{}
% use microtype if available
\IfFileExists{microtype.sty}{\usepackage{microtype}}{}

% Table formatting
\usepackage{longtable, booktabs}
\usepackage{lscape}
% \usepackage[counterclockwise]{rotating}   % Landscape page setup for large tables
\usepackage{multirow}		% Table styling
\usepackage{tabularx}		% Control Column width
\usepackage[flushleft]{threeparttable}	% Allows for three part tables with a specified notes section
\usepackage{threeparttablex}            % Lets threeparttable work with longtable

% Create new environments so endfloat can handle them
% \newenvironment{ltable}
%   {\begin{landscape}\begin{center}\begin{threeparttable}}
%   {\end{threeparttable}\end{center}\end{landscape}}

\newenvironment{lltable}
  {\begin{landscape}\begin{center}\begin{ThreePartTable}}
  {\end{ThreePartTable}\end{center}\end{landscape}}

  \usepackage{ifthen} % Only add declarations when endfloat package is loaded
  \ifthenelse{\equal{\string man}{\string man}}{%
   \DeclareDelayedFloatFlavor{ThreePartTable}{table} % Make endfloat play with longtable
   % \DeclareDelayedFloatFlavor{ltable}{table} % Make endfloat play with lscape
   \DeclareDelayedFloatFlavor{lltable}{table} % Make endfloat play with lscape & longtable
  }{}%



% The following enables adjusting longtable caption width to table width
% Solution found at http://golatex.de/longtable-mit-caption-so-breit-wie-die-tabelle-t15767.html
\makeatletter
\newcommand\LastLTentrywidth{1em}
\newlength\longtablewidth
\setlength{\longtablewidth}{1in}
\newcommand\getlongtablewidth{%
 \begingroup
  \ifcsname LT@\roman{LT@tables}\endcsname
  \global\longtablewidth=0pt
  \renewcommand\LT@entry[2]{\global\advance\longtablewidth by ##2\relax\gdef\LastLTentrywidth{##2}}%
  \@nameuse{LT@\roman{LT@tables}}%
  \fi
\endgroup}


\ifxetex
  \usepackage[setpagesize=false, % page size defined by xetex
              unicode=false, % unicode breaks when used with xetex
              xetex]{hyperref}
\else
  \usepackage[unicode=true]{hyperref}
\fi
\hypersetup{breaklinks=true,
            pdfauthor={},
            pdftitle={The title},
            colorlinks=true,
            citecolor=blue,
            urlcolor=blue,
            linkcolor=black,
            pdfborder={0 0 0}}
\urlstyle{same}  % don't use monospace font for urls

\setlength{\parindent}{0pt}
%\setlength{\parskip}{0pt plus 0pt minus 0pt}

\setlength{\emergencystretch}{3em}  % prevent overfull lines

\ifxetex
  \usepackage{polyglossia}
  \setmainlanguage{}
\else
  \usepackage[english]{babel}
\fi

% Manuscript styling
\captionsetup{font=singlespacing,justification=justified}
\usepackage{csquotes}
\usepackage{upgreek}



\usepackage{tikz} % Variable definition to generate author note

% fix for \tightlist problem in pandoc 1.14
\providecommand{\tightlist}{%
  \setlength{\itemsep}{0pt}\setlength{\parskip}{0pt}}

% Essential manuscript parts
  \title{The title}

  \shorttitle{Expressive Writing}


  \author{First Author\textsuperscript{1}~\& author name\textsuperscript{1,2}}

  \def\affdep{{"", ""}}%
  \def\affcity{{"", ""}}%

  \affiliation{
    \vspace{0.5cm}
          \textsuperscript{1} institution1\\
          \textsuperscript{2} institution2  }

  \authornote{
    \newcounter{author}
    Add complete departmental affiliations for each author here. Each new
    line herein must be indented, like this line.
    
    Enter author note here.

                      Correspondence concerning this article should be addressed to First Author, Postal address. E-mail: \href{mailto:my@email.com}{\nolinkurl{my@email.com}}
                          }


  \abstract{Enter abstract here. Each new line herein must be indented, like this
line.}
  \keywords{keywords \\

    \indent Word count: X
  }





\usepackage{amsthm}
\newtheorem{theorem}{Theorem}
\newtheorem{lemma}{Lemma}
\theoremstyle{definition}
\newtheorem{definition}{Definition}
\newtheorem{corollary}{Corollary}
\newtheorem{proposition}{Proposition}
\theoremstyle{definition}
\newtheorem{example}{Example}
\theoremstyle{definition}
\newtheorem{exercise}{Exercise}
\theoremstyle{remark}
\newtheorem*{remark}{Remark}
\newtheorem*{solution}{Solution}
\begin{document}

\maketitle

\setcounter{secnumdepth}{0}



\newpage

\setcounter{page}{1}

\section{Meta-Analysis References}\label{meta-analysis-references}

\setlength{\parindent}{-0.5in} \setlength{\leftskip}{0.5in}

\hypertarget{refs}{}
\hypertarget{ref-Arden-Close2013}{}
Arden-Close, E., Gidron, Y., Bayne, L., \& Moss-Morris, R. (2013).
Written emotional disclosure for women with ovarian cancer and their
partners: randomised controlled trial. \emph{Psycho-Oncology},
\emph{22}(10), 2262--2269.
doi:\href{https://doi.org/10.1002/pon.3280}{10.1002/pon.3280}

\hypertarget{ref-Barry2001}{}
Barry, L. M., \& Singer, G. H. (2001). Reducing maternal psychological
distress after the NICU experience through journal writing.
\emph{Journal of Early Intervention}, \emph{24}(4), 287--297.
doi:\href{https://doi.org/10.1177/105381510102400404}{10.1177/105381510102400404}

\hypertarget{ref-Broderick2005}{}
Broderick, J. E., Junghaenel, D. U., \& Schwartz, J. E. (2005). Written
emotional expression produces health benefits in Fibromyalgia patients.
\emph{Psychosomatic Medicine}, \emph{67}(2), 326--334.
doi:\href{https://doi.org/10.1097/01.psy.0000156933.04566.bd}{10.1097/01.psy.0000156933.04566.bd}

\hypertarget{ref-Bugg2009}{}
Bugg, A., Turpin, G., Mason, S., \& Scholes, C. (2009). A randomised
controlled trial of the effectiveness of writing as a self-help
intervention for traumatic injury patients at risk of developing
post-traumatic stress disorder. \emph{Behaviour Research and Therapy},
\emph{47}(1), 6--12.
doi:\href{https://doi.org/10.1016/j.brat.2008.10.006}{10.1016/j.brat.2008.10.006}

\hypertarget{ref-Craft2013}{}
Craft, M. A., Davis, G. C., \& Paulson, R. M. (2013). Expressive writing
in early breast cancer survivors. \emph{Journal of Advanced Nursing},
\emph{69}(2), 305--315.
doi:\href{https://doi.org/10.1111/j.1365-2648.2012.06008.x}{10.1111/j.1365-2648.2012.06008.x}

\hypertarget{ref-Deters2003}{}
Deters, P. B., \& Range, L. M. (2003). Does writing reduce posttraumatic
stress disorder symptoms? \emph{Violence and Victims}, \emph{18}(5),
569--580.
doi:\href{https://doi.org/10.1891/vivi.2003.18.5.569}{10.1891/vivi.2003.18.5.569}

\hypertarget{ref-Blasio2015a}{}
Di Blasio, P., Camisasca, E., Caravita, S. C. S., Ionio, C., Milani, L.,
Valtolina, G. G., \ldots{} Valtolina, G. G. (2015). The effects of
expressive writing on postpartum depression and posttraumatic stress
symptoms. \emph{Psychological Reports}, \emph{117}(3), 856--882.
doi:\href{https://doi.org/10.2466/02.13.PR0.117c29z3}{10.2466/02.13.PR0.117c29z3}

\hypertarget{ref-Gebler2007}{}
Gebler, F. A., \& Maercker, A. (2007). Expressive writing and
existential writing as coping with traumatic experiences -- A randomized
controlled pilot study. \emph{Trauma \& Gewalt}, \emph{1}(1), 264--272.

\hypertarget{ref-Gellaitry2010}{}
Gellaitry, G., Peters, K., Bloomfield, D., \& Horne, R. (2010).
Narrowing the gap: The effects of an expressive writing intervention on
perceptions of actual and ideal emotional support in women who have
completed treatment for early stage breast cancer.
\emph{Psycho-Oncology}, \emph{19}(1), 77--84.
doi:\href{https://doi.org/10.1002/pon.1532}{10.1002/pon.1532}

\hypertarget{ref-Giannotta2009}{}
Giannotta, F., Settanni, M., Kliewer, W., \& Ciairano, S. (2009).
Results of an Italian school-based expressive writing intervention trial
focused on peer problems. \emph{Journal of Adolescence}, \emph{32}(6),
1377--1389.
doi:\href{https://doi.org/10.1016/j.adolescence.2009.07.001}{10.1016/j.adolescence.2009.07.001}

\hypertarget{ref-Gidron1996a}{}
Gidron, Y., Peri, T., Connolly, J. F., \& Shalev, A. Y. (1996). Written
disclosure in posttraumatic stress disorder: Is it beneficial for the
patient? \emph{The Journal of Nervous and Mental Disease},
\emph{184}(8), 505--506.
doi:\href{https://doi.org/10.1097/00005053-199608000-00009}{10.1097/00005053-199608000-00009}

\hypertarget{ref-Greenberg1996}{}
Greenberg, M. A., Wortman, C. B., \& Stone, A. A. (1996). Emotional
expression and physical health: Revising traumatic memories or fostering
self-regulation? \emph{Journal of Personality and Social Psychology},
\emph{71}(3), 588--602.
doi:\href{https://doi.org/10.1037/0022-3514.71.3.588}{10.1037/0022-3514.71.3.588}

\hypertarget{ref-Halpert2010}{}
Halpert, A., Rybin, D., \& Doros, G. (2010). Expressive writing is a
promising therapeutic modality for the management of IBS: A pilot study.
\emph{The American Journal of Gastroenterology}, \emph{105}(11),
2440--2448.
doi:\href{https://doi.org/10.1038/ajg.2010.246}{10.1038/ajg.2010.246}

\hypertarget{ref-Horsch2016}{}
Horsch, A., Tolsa, J.-F., Gilbert, L., Chêne, L. J. du, Müller-Nix, C.,
\& Graz, M. B. (2016). Improving maternal mental health following
preterm birth using an expressive writing intervention: A randomized
controlled trial. \emph{Child Psychiatry \& Human Development},
\emph{47}(5), 780--791.
doi:\href{https://doi.org/10.1007/s10578-015-0611-6}{10.1007/s10578-015-0611-6}

\hypertarget{ref-Hoyt2014}{}
Hoyt, T., \& Renshaw, K. D. (2014). Emotional disclosure and
posttraumatic stress symptoms: Veteran and spouse reports.
\emph{International Journal of Stress Management}, \emph{21}(2),
186--206. doi:\href{https://doi.org/10.1037/a0035162}{10.1037/a0035162}

\hypertarget{ref-Hoyt2011}{}
Hoyt, T., \& Yeater, E. A. (2011). The effects of negative emotion and
expressive writing on posttraumatic stress symptoms. \emph{Journal of
Social and Clinical Psychology}, \emph{30}(6), 549--569.
doi:\href{https://doi.org/10.1521/jscp.2011.30.6.549}{10.1521/jscp.2011.30.6.549}

\hypertarget{ref-Ironson2013}{}
Ironson, G., O'Cleirigh, C., Leserman, J., Stuetzle, R., Fordiani, J.,
Fletcher, M., \& Schneiderman, N. (2013). Gender-specific effects of an
augmented written emotional disclosure intervention on posttraumatic,
depressive, and HIV-disease-related outcomes: A randomized, controlled
trial. \emph{Journal of Consulting and Clinical Psychology},
\emph{81}(2), 284--298.
doi:\href{https://doi.org/10.1037/a0030814}{10.1037/a0030814}

\hypertarget{ref-Kallay2008}{}
Kállay, É., \& Băban, A. (2008). Emotional benefits of expressive
writing in a sample of Romanian female cancer patients. \emph{Cognition
Brain Behavior}, \emph{12}(1), 115--129.

\hypertarget{ref-Knaevelsrud2007}{}
Knaevelsrud, C., \& Maercker, A. (2007). Internet-based treatment for
PTSD reduces distress and facilitates the development of a strong
therapeutic alliance: A randomized controlled clinical trial. \emph{BMC
Psychiatry}, \emph{7}(1), 13.
doi:\href{https://doi.org/10.1186/1471-244X-7-13}{10.1186/1471-244X-7-13}

\hypertarget{ref-Koopman2005}{}
Koopman, C., Ismailji, T., Holmes, D., Classen, C. C., Palesh, O., \&
Wales, T. (2005). The effects of expressive writing on pain, depression
and posttraumatic stress disorder symptoms in survivors of intimate
partner violence. \emph{Journal of Health Psychology}, \emph{10}(2),
211--221.
doi:\href{https://doi.org/10.1177/1359105305049769}{10.1177/1359105305049769}

\hypertarget{ref-Kovac2000}{}
Kovac, S. H., \& Range, L. M. (2000). Writing projects: Lessening
undergraduates' unique suicidal bereavement. \emph{Suicide \&
Life-Threatening Behavior}, \emph{30}(1), 50--60.

\hypertarget{ref-Lancaster2015}{}
Lancaster, S. L., Klein, K. P., \& Heifner, A. (2015). The validity of
self-reported growth after expressive writing. \emph{Traumatology},
\emph{21}(4), 293--298.
doi:\href{https://doi.org/10.1037/trm0000052}{10.1037/trm0000052}

\hypertarget{ref-Lange2001}{}
Lange, A., van de Ven, J.-P., Schrieken, B., \& Emmelkamp, P. M. (2001).
Interapy. Treatment of posttraumatic stress through the Internet: A
controlled trial. \emph{Journal of Behavior Therapy and Experimental
Psychiatry}, \emph{32}(2), 73--90.
doi:\href{https://doi.org/10.1016/S0005-7916(01)00023-4}{10.1016/S0005-7916(01)00023-4}

\hypertarget{ref-Low2010}{}
Low, C. A., Stanton, A. L., Bower, J. E., \& Gyllenhammer, L. (2010). A
randomized controlled trial of emotionally expressive writing for women
with metastatic breast cancer. \emph{Health Psychology}, \emph{29}(4),
460--466. doi:\href{https://doi.org/10.1037/a0020153}{10.1037/a0020153}

\hypertarget{ref-Lu2012a}{}
Lu, Q., Zheng, D., Young, L., Kagawa-Singer, M., \& Loh, A. (2012). A
pilot study of expressive writing intervention among Chinese-speaking
breast cancer survivors. \emph{Health Psychology}, \emph{31}(5),
548--551. doi:\href{https://doi.org/10.1037/a0026834}{10.1037/a0026834}

\hypertarget{ref-Meshberg-Cohen2014}{}
Meshberg-Cohen, S., Svikis, D., \& McMahon, T. J. (2014). Expressive
writing as a therapeutic process for drug-dependent women.
\emph{Substance Abuse}, \emph{35}(1), 80--88.
doi:\href{https://doi.org/10.1080/08897077.2013.805181}{10.1080/08897077.2013.805181}

\hypertarget{ref-Nixon2009}{}
Nixon, R. D. V., \& Kling, L. W. (2009). Treatment of adult
post-traumatic stress disorder using a future-oriented writing therapy
approach. \emph{The Cognitive Behaviour Therapist}, \emph{2}(04),
243--255.
doi:\href{https://doi.org/10.1017/S1754470X09990171}{10.1017/S1754470X09990171}

\hypertarget{ref-Park2002}{}
Park, C. L., \& Blumberg, C. J. (2002). Disclosing trauma through
writing: Testing the meaning-making hypothesis. \emph{Cognitive Therapy
and Research}, \emph{26}(5), 597--616.
doi:\href{https://doi.org/10.1023/A:1020353109229}{10.1023/A:1020353109229}

\hypertarget{ref-Paez1999}{}
Páez, D., Velasco, C., \& González, J. L. (1999). Expressive writing and
the role of Alexythimia as a dispositional deficit in self-disclosure
and psychological health. \emph{Journal of Personality and Social
Psychology}, \emph{77}(3), 630--641.
doi:\href{https://doi.org/10.1037/0022-3514.77.3.630}{10.1037/0022-3514.77.3.630}

\hypertarget{ref-Possemato2010}{}
Possemato, K., Ouimette, P., \& Geller, P. (2010). Internet-based
expressive writing for kidney transplant recipients: Effects on
posttraumatic stress and quality of life. \emph{Traumatology},
\emph{16}(1), 49--54.
doi:\href{https://doi.org/10.1177/1534765609347545}{10.1177/1534765609347545}

\hypertarget{ref-Range2000}{}
Range, L. M., Kovac, S. H., \& Marion, M. S. (2000). Does writing about
the bereavement lessen grief following sudden, unintentional death?
\emph{Death Studies}, \emph{24}, 115--134.
doi:\href{https://doi.org/10.1080/074811800200603}{10.1080/074811800200603}

\hypertarget{ref-Schoutrop2002}{}
Schoutrop, M. J. A., Lange, A., Hanewald, G., Davidovich, U., \&
Salomon, H. H. (2002). Structured writing and processing major stressful
events: A controlled trial. \emph{Psychotherapy and Psychosomatics},
\emph{71}(3), 151--157.
doi:\href{https://doi.org/10.1159/000056282}{10.1159/000056282}

\hypertarget{ref-Schwartz2004}{}
Schwartz, L. (2004). Effects of written emotional disclosure on
caregivers of children and adolescents with chronic illness.
\emph{Journal of Pediatric Psychology}, \emph{29}(2), 105--118.
doi:\href{https://doi.org/10.1093/jpepsy/jsh014}{10.1093/jpepsy/jsh014}

\hypertarget{ref-Slavin-Spenny2011}{}
Slavin-Spenny, O. M., Cohen, J. L., Oberleitner, L. M., \& Lumley, M. A.
(2011). The effects of different methods of emotional disclosure:
Differentiating posttraumatic growth from stress symptoms. \emph{Journal
of Clinical Psychology}, \emph{67}(10), 993--1007.
doi:\href{https://doi.org/10.1002/jclp.20750}{10.1002/jclp.20750}

\hypertarget{ref-Sloan2004}{}
Sloan, D. M., \& Marx, B. P. (2004). A closer examination of the
structured written disclosure procedure. \emph{Journal of Consulting and
Clinical Psychology}, \emph{72}(2), 165--175.
doi:\href{https://doi.org/10.1037/0022-006X.72.2.165}{10.1037/0022-006X.72.2.165}

\hypertarget{ref-Sloan2005}{}
Sloan, D. M., Marx, B. P., \& Epstein, E. M. (2005). Further examination
of the exposure model underlying the efficacy of written emotional
disclosure. \emph{Journal of Consulting and Clinical Psychology},
\emph{73}(3), 549--554.
doi:\href{https://doi.org/10.1037/0022-006X.73.3.549}{10.1037/0022-006X.73.3.549}

\hypertarget{ref-Sloan2011a}{}
Sloan, D. M., Marx, B. P., \& Greenberg, E. M. (2011). A test of written
emotional disclosure as an intervention for posttraumatic stress
disorder. \emph{Behaviour Research and Therapy}, \emph{49}(4), 299--304.
doi:\href{https://doi.org/10.1016/j.brat.2011.02.001}{10.1016/j.brat.2011.02.001}

\hypertarget{ref-Smyth2002a}{}
Smyth, J. M., Anderson, C. F., Hockemeyer, J. R., \& Stone, A. A.
(2002). Does emotional non-expressiveness or avoidance interfere with
writing about stressful life events? An analysis in patients with
chronic illness. \emph{Psychology \& Health}, \emph{17}(5), 561--569.
doi:\href{https://doi.org/10.1080/08870440290025777}{10.1080/08870440290025777}

\hypertarget{ref-Smyth2002}{}
Smyth, J. M., Hockemeyer, J., Anderson, C., Strandberg, K., Koch, M.,
O'Neill, H. K., \& McCammon, S. (2002). Structured writing about a
natural disaster buffers the effect of intrusive thoughts on negative
affect and physical symptoms. \emph{Australasian Journal of Disaster and
Trauma Studies}, \emph{6}(1).

\hypertarget{ref-Stockton2014}{}
Stockton, H., Joseph, S., \& Hunt, N. (2014). Expressive writing and
posttraumatic growth: An internet-based study. \emph{Traumatology: An
International Journal}, \emph{20}(2), 75--83.
doi:\href{https://doi.org/10.1037/h0099377}{10.1037/h0099377}

\hypertarget{ref-Ullrich2002a}{}
Ullrich, P. M., \& Lutgendorf, S. K. (2002). Journaling about stressful
events: Effects of cognitive processing and emotional expression.
\emph{Annals of Behavioral Medicine}, \emph{24}(3), 244--250.
doi:\href{https://doi.org/10.1207/S15324796ABM2403_10}{10.1207/S15324796ABM2403\_10}

\hypertarget{ref-Vedhara2007}{}
Vedhara, K., Morris, R. M., Booth, R., Horgan, M., Lawrence, M., \&
Birchall, N. (2007). Changes in mood predict disease activity and
quality of life in patients with psoriasis following emotional
disclosure. \emph{Journal of Psychosomatic Research}, \emph{62}(6),
611--619.
doi:\href{https://doi.org/10.1016/j.jpsychores.2006.12.017}{10.1016/j.jpsychores.2006.12.017}

\hypertarget{ref-Wagner2006}{}
Wagner, B., Knaevelsrud, C., \& Maercker, A. (2006). Internet-based
cognitive-behavioral therapy for complicated grief: A randomized
controlled trial. \emph{Death Studies}, \emph{30}(5), 429--453.
doi:\href{https://doi.org/10.1080/07481180600614385}{10.1080/07481180600614385}

\hypertarget{ref-Walker1999a}{}
Walker, B. L., Nail, L. M., \& Croyle, R. T. (1999). Does emotional
expression make a difference in reactions to breast cancer?
\emph{Oncology Nursing Forum}, \emph{26}(6), 1025--1032.

\hypertarget{ref-Zakowski2004}{}
Zakowski, S. G., Ramati, A., Johnson, P., Flanigan, R., Morton, C.,
Johnson, P., \& Flanigan, R. (2004). Written emotional disclosure
buffers the effects of social constraints on distress among cancer
patients. \emph{Health Psychology}, \emph{23}(6), 555--563.
doi:\href{https://doi.org/10.1037/0278-6133.23.6.555}{10.1037/0278-6133.23.6.555}






\end{document}
